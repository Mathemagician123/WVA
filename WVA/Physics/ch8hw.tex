\documentclass[11pt]{scrartcl}
\usepackage[utf8]{inputenc}
\usepackage{amsmath, amssymb, amsthm, bbm}
\usepackage{booktabs, verbatim, graphicx, framed}
\usepackage[sexy, hints]{evan}
\title{Chapter 8 Homework}
\author{Anay Aggarwal}

\begin{document}

\maketitle
\begin{example}
  Clay problem, part a.
\end{example}
\begin{soln}
  To solve this problem, we need to know about conservation of
  angular momentum. We need to know that $L=I\omega$, and we will
  need to know that the moment of inertia of a solid disk is $\frac{1}{2}MR^2$,
  where $M$ is the mass and $R$ is the radius (I proved this in the last homework).
  \\ \\
  Let $M$ be the mass of the potters wheel, $R$ be the radius of the potters wheel,
  $m$ be the mass of the disk of clay, $r$ be the radius of the disk of clay. Let $\omega_i$
  be the angular velocity of the wheel initially, and $\omega_f$ be the angular velocity of the
  wheel finally.
  \\ \\
  The only force applied to the wheel is vertically downwards, so it doesn't effect
  the disks spinning about the vertical axis, so there is no net torque. Hence, be conservation
  of angular momentum,
  $$\Delta L=\tau \cdot t = 0$$
  $$L_i=L_f$$
  $$I_i \omega_i = I_f \omega_f$$
  $$\frac{1}{2}MR^2 \omega_i=I_f \omega_f$$
  In order to compute $\omega_f$, it remains to compute $I_f$. Note that the final
  moment of inertia is the initial moment plus the moment added on.
  Since the clay added is a disk, the added moment of inertia is $\frac{1}{2}mr^2$.
  Thus
  $$\frac{1}{2}MR^2 \omega_i=\left(\frac12 MR^2+\frac12 mr^2\right)\omega_f$$
  $$\omega_f=\omega_i\left(\frac{MR^2}{MR^2+mr^2}\right)$$
  Which how fast the wheel is now spinning!
  \\ \\
  Some symbolic analysis: The units are correct since $\frac{MR^2}{MR^2+mr^2}$ is a
  scalar (the numerator and the denominator are both $kg m^2$). If the clay disk is
  almost nothing to the potters wheel, the final angular velocity will almost be
  the same as the initial angular velocity, which makes sense.
  \\ \\
  Let's plug in the numbers:
  $$\omega_f=(1.5 rev/s)\left(\frac{5(0.2)^2 kg m^2}{5(0.2)^2 + 3.1(0.04)^2 kg m^2}\right)$$
  $$\omega_f\approx (1.5 rev/s)(0.98)=1.47 rev/s$$
  This makes sense, it's just a little bit under the original velocity since the clay
  is very little.
\end{soln}
\begin{example}
  Clay problem, part b.
\end{example}
\begin{soln}
  This problem is nearly analogous to the previous problem, except that the
  final moment of inertia is different because the clay is a point mass and it is dropped at
  the edge of the disk. Hence we will use everything the same in the previous solution,
  up until we computed $I_f$. We have that
  $$\frac{1}{2}MR^2 \omega_i=I_f \omega_f$$
  Now, $I_f$ is the sum of the initial moment and the added moment, as before.
  This time, the added moment is of a point mass, so it's moment is $mR^2$.
  This is because it has a mass $m$ and is a distance $R$ (notice, big $R$ not small $r$)
  away from the axis that it's rotating around. Thus
  $$\frac{1}{2}MR^2 \omega_i=\left(\frac{1}{2}MR^2+mR^2\right)\omega_f$$
  $$M \omega_i = (M+2m)\omega_f$$
  $$\omega_f=\omega_i\left(\frac{M}{M+2m}\right)$$
  Which is the answer!
  \\ \\
  Some symbolic check: the units are correct because $\frac{M}{M+2m}$ is a scalar,
  since the denominator and numerator both have units $kg$. As $M$ is large compared to $m$,
  $\omega_f$ becomes closer to $\omega_i$ as expected. As $m$ is large compared to $M$,
  $\omega_f$ becomes far less than $\omega_i$, which makes sense. $R$ doesn't matter,
  which makes sense because the clay is just a point mass.
  \\ \\
  Now to plug in the numbers:
  $$\omega_f=(1.5 rev/s)\left(\frac{5 kg}{5kg+6.2kg}\right)\approx 0.67 rev/s$$
  This makes sense because $\omega_f$ is significantly less than $\omega_i$, because the clay is heavy.
\end{soln}
\begin{example}
  Clay problem, part c.
\end{example}
\begin{soln}
  To solve this, we need to know that rotational energy is $\frac{1}{2}I\omega^2$.
  Hence the change in rotational energy is
  $$\frac{1}{2}I_f\omega_f^2-\frac{1}{2}I_i\omega_i^2$$
  We can simply plug in $I_f, I_i, \omega_f, \omega_i$ for each case.
  \\ \\
  Case 1: Part a. We can use the values from part a to get:
  $$\Delta \mathrm{RE}=\frac{1}{2}I_f\omega_f^2-\frac{1}{2}I_i\omega_i^2$$
  $$\Delta \mathrm{RE}=0.5((0.102 kg m^2)(1.47 rev/s)^2-(0.1 kg m^2)(1.5 rev/s)^2)$$
  $$\Delta \mathrm{RE}=-0.002 kg m^2 rev^2/s^2$$
  Notice that we lost some energy! This is probably due to heat and sound.
  The clay falling onto the wheel makes a sound and exerts a bit of heat, which
  is where the energy is lost.
  \\ \\
  Case 2: Part b. We can use the values from part b to get:
  $$\Delta \mathrm{RE}=\frac{1}{2}I_f \omega_f^2-\frac{1}{2}I_i\omega_i^2$$
  $$\Delta \mathrm{RE}=\frac{1}{2}(0.162 kgm^2)(0.67 rev/s)^2-\frac{1}{2}(0.1 kgm^2)(1.5 rev/s)^2$$
  $$\Delta \mathrm{RE}=-0.076 kg m^2 rev^2/s^2$$
  We lost some energy here too, due to the same reasons as before. We lost
  more energy here because the blob is now a point-mass so we lost more velocity
  (as shown by the results in parts a and b).
\end{soln}
\end{document}
