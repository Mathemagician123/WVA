\documentclass[11pt]{scrartcl}
\usepackage[utf8]{inputenc}
\usepackage{amsmath, amssymb, amsthm, bbm}
\usepackage{booktabs, verbatim, graphicx, framed}
\usepackage[sexy, hints]{evan}
\title{Chapter 24 Homework}
\author{Anay Aggarwal}

\begin{document}

\maketitle
\begin{example}
  [Question 3]
  Why is light sometimes described as a ray and sometimes a wave?
\end{example}
\begin{soln}
  Because it acts like a wave - it diffracts, but it also reflects like a ray.
\end{soln}
\begin{example}
  [Question 4]
  Why can't you see around corners even though you can hear?
\end{example}
\begin{soln}
  Because light cannot bend around a corner. The wavelength of light is so small that it cannot diffract around the door.
\end{soln}
\begin{example}
  [Question 6]
  Red light vs Blue light
\end{example}
\begin{soln}
  The wavelength of red light is different from the wavelength of blue light, it's higher (as the frequency of red light is lower). To figure out where the bright fringes
  are, we can use the equation
  $$\frac{dy_m}{L} = m\lambda$$
  $$y_m=\frac{m\lambda L}{d}$$
  So when $\lambda$ is smaller, $y_m$ (the distance between the fringe and the central fringe) becomes smaller. So the fringes get closer together.
\end{soln}
\begin{example}
  [Question 7]
  Two rays of light from the same source destructively interfere if their path lengths differ by how much?
\end{example}
\begin{soln}
  $\frac{\lambda}{2}$, half the wavelength.
\end{soln}
\begin{example}
  [Problem 36]
  Diffraction grating.
\end{example}
\begin{soln}
  Suppose there are $x=8500$ lines per centimer. To solve this problem, we need to know that the distance $d$ for the diffraction grating is $\frac{1}{x}$. We also need to use that
  $$\frac{dy_m}{L}=m\lambda$$
  $$y_m=\frac{m\lambda L}{d}=m\lambda L x$$
  We are looking at the first order spectrum ($m=1$), so
  $$y_1=\lambda L x$$
  But since we have white light, the wavelength is in the interval $[\lambda_1, \lambda_2]$, where $\lambda_1=410 \mathrm{nm}, \lambda_2=750 \mathrm{nm}$.
  Thus the width of the fringe is the difference between the two $y$'s for $\lambda_2$ and $\lambda_1$, i.e. we desire the quantity
  $$(\lambda_2-\lambda_1)Lx$$
  Plugging the numbers in,
  $$(340\mathrm{nm})(2.3 \mathrm{m})(8500 \frac{1}{\mathrm{cm}})=665\mathrm{mm}$$
  Which indeed has the correct units and makes sense.
\end{soln}
\begin{example}
  [Problem 66]
  Radio signal
\end{example}
\begin{soln}
  In order to find where the minimum and maximum signals are, we can use the equations:
  $$d\sin\theta=m\lambda\implies \sin\theta=\frac{m\lambda}{d}$$
  $$d\sin\theta=\left(m+\frac{1}{2}\right)\lambda \implies \sin\theta=\frac{\left(m+\frac{1}{2}\right)}{d}$$
  Where the first equation is for constructive interference (maximum) and the second is for destructive (minimum).
  We are given the frequency $f$, so we can use $\lambda = \frac{v}{f}=\frac{c}{f}$, since the velocity of the wave
  is just the speed of light.
  Thus when we want maximum signal,
  $$\theta=\arcsin\left(\frac{mc}{df}\right)$$
  Where $m$ is an integer. Note that when $m=0$, we get the given $\theta=0^{\circ}$.
  And when we want a minimum signal,
  $$\theta=\arcsin\left(\frac{\left(m+\frac{1}{2}\right)c}{df}\right)$$
  Where $m$ is an integer. All other quantities are given in the problem.
\end{soln}
\end{document}
